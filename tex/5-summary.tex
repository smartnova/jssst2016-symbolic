\section {まとめと今後の課題}
\label {sec: summary}

本稿では数学的知識を背景とする領域のソフトウェア記述とそのドキュメントの内部記述について扱い,その困難が問題領域であるところの数学の世界とプログラミングをする計算の世界の間の大きなギャップにあること,そして,このギャップを埋めるべき内部記述を助けるツールが存在しないことを指摘した.

この問題について,数学の世界を数式処理システムによって記述し,数式から自動生成したプログラム断片や\LaTeX{}書式の文書断片をプログラムや内部記述に埋め込むことによって,問題をきれいに解消できることを示した.

今後は,この技術の適用範囲をさらに拡大するとともに,ソフトウェアドキュメントの記述上の問題をさらに深く検討したい.

\textbf {謝辞}\
この論文の執筆にあたって,時間をさいて議論に応じて下さった池谷のぞみ先生と権藤克彦先生に感謝します.本稿のアイデアについて萌芽的な段階でご議論をいただいたSIGPX参加者と、特に参考文献をご教示下さった,加藤淳先生,増原英彦先生,渡部卓雄先生に感謝します.
