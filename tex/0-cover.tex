\title {$^*$数学的記号処理システムを用いたソフトウェアの構成手法
\thanks {This is an unrefereed paper.  Copyrights belong to the authors.}}

\author{脇田建 高野陸 武田一起
%
% ここにタイトル英訳 (英文の場合は和訳) を書く.
%
% \ejtitle {A Software Construction Method Using Symbolic Algebraic Computation}
%
% ここに著者英文表記 (英文の場合は和文表記) および
% 所属 (和文および英文) を書く.
% 複数著者の所属はまとめてよい.
%
\thanks{脇田建、東京工業大学情報理工学院数理・計算科学系、CREST/JST、%
高野陸、武田一起、東京工業大学理学部情報科学科}}

% 和文アブストラクト
\Jabstract{%
ソフトウェアの構築、継続的な保守、文書管理は煩雑な作業を伴うソフトウェア工学上のさまざまな問題を孕んでいる。本研究は特に数学的な理論を応用するソフトウェアシステムをとりあげ、簡素な数学的構成とそれを実装したソースコードの複雑さに根差す概念的なギャップの問題、このことに付随する文書執筆と保守の困難を研究対象とする。本稿では、この問題について数学的記号処理システムを系統的に応用する方法を提案する。ソフトウェアの核心部となる数学的概念は記号システムによって記述し、それからソフトウェア、文書、文書に付随する図等を自動合成することによって、さまざまな困難を解消できることを示す。事例としては、各種グラフ可視化アルゴリズムとその対話システム、3次元グラフィックス用の空間変換ライブラリなどを取り上げ、従来の実装と実装の複雑さ、性能、システムの可読性、保守性などを論じる。}

% 英文アブストラクト(本サンプルの原論文にはなし)
% \Eabstract{ }
%
\maketitle \thispagestyle {empty}
